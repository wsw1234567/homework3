\documentclass{ctexart}
\usepackage{xltxtra}
\usepackage{url}


\title{我的Linux工作环境}
\author{吴声炜\\专业:数学与应用数学\\学号:3210102945}
\begin{document}

\section{我的Linux基本信息}

我所安装的\verb|Linux|是18.04版本的\verb|ubuntu|的虚拟机。

\section{Linux的近一步调整}

在安装完成虚拟机之后,我安装了包管理器\verb|synaptic|,在上面安装了诸如\verb|g++|,\verb|gcc|,\verb|make|,\verb|cmake|,\verb|automake|,\verb|emacs|,\verb|vim|,\verb|texlive|的软件,其中有些已经学习使用了,有些还没有用到。作为一个中国学生,我肯定会为我的系统安装中文输入法,并且还为\verb|emacs|设置了个性化的配置文件,比如调整背景颜色,支持输入中文之类的。此外还将本地代码库上传到了\verb|github|中,更好地保存代码,以防出现意外(虚拟机崩溃)导致代码文件丢失。

\section{下一步的规划}

短期内的规划是在数学软件这门短学期课程上学习\verb|Linux|的更多知识,以及练习熟悉它的使用。
长期来看,需要在学习中更加频繁地使用\verb|Linux|,适应以后可能的工作环境。

\section{对未来的估计}
\subsection{未来半年使用Linux的场合}

由于下学期我并没有有关信息与计算科学的课程,应该是不会在数学学习中使用到\verb|Linxu|环境,但其他的通识课程的论文或许可以使用\verb|texlive|来写,可以显得更加专业也说不定。

\subsection{是否符合未来需求}

由于我还是只是大一,对之后的课程了解不是很多,我猜测之后的学习中,需要我保存的代码,文件有很多,但是我现在的文件保存还是比较混乱。
因此,为了更加优雅地使用保存代码,我将学习使用\verb|github|进行统一地代码管理\cite{Git教程-廖雪峰的官方网站}。

\section{如何保证工作系统中的代码,文献和工作结果的稳定和安全}

我认为应该需要保持不同代码之间的独立性,不能有互相冲突的代码,同时又需要在编写上使用统一规范。使用\verb|github|使得代码修改可查询,可撤销。同时也减少文件在本地的复制,可能导致之后工作的混乱。

\bibliographystyle{plain}
\bibliography{bibfile}


\end{document}
